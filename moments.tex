\documentclass{article}
\usepackage{graphicx}
\usepackage{amsmath}
\usepackage{amssymb}
\usepackage{float}
\usepackage{cleveref}
\usepackage[multidot]{grffile}  
\title{Variance exploration}
% \author{aaghaeeyan }
% \date{September 2025}

\begin{document}
	\section{Expected case reproductive number $R_c$}	
	\subsection{A model with a more general incidence term}
Assume that the following relation between the incidence, $i$, and the expected reproductive number is true
	\begin{subequations} 
		\label{eq_CaseReproductiveNumber}
		\begin{align}
	C(\tau, \rho) &= \int_{t = \tau}^{t= \rho} h(t) dt,\\
	\omega (\tau, \rho) & = i(\tau) f(\rho - \tau),\\
	i(t) & = h(t) y(t),
	\end{align}
	\end{subequations}
	where $h(t)$ is some function of states and, in turn, time, and
	$y(t)$ is the proportion of infectious individuals.
	
	If so, it appears that the functional form of $h(t)$ does not matter. Please see below. 
	
	The second raw moment of the expected case reproductive number $C(\tau,\rho)$ reads as follows
	\begin{small}
		\begin{subequations}
		\begin{align} 
		C_2 &= \int_{\tau = 0}^\infty \int_{\rho = \tau}^\infty i(\tau) f(\rho-\tau) d\rho d\tau \int_{t=\tau}^{\rho} \int_{s=\tau}^{\rho} h(t) h(s) dt ds\\
		&= 2 \int_{\tau = 0}^\infty \int_{\rho = \tau}^\infty 
		\int_{t=s}^{\rho} \int_{s=\tau}^{t}i(\tau) f(\rho-\tau) d\rho d\tau  h(t) h(s) dt ds\\
		& = 2 \int_{\tau = 0}^\infty \int_{t=\tau}^\infty \int_{s=\tau}^t i(\tau)h(t)h(s)ds \underbrace{\int_{\rho = t}^\infty f(\rho - \tau) d\rho}_{F(t-\tau)} \label{eq_before_Markovian}\\
		& = 2 \int_{\tau = 0}^\infty \int_{t=\tau}^\infty \int_{s=\tau}^t i(\tau)h(t)h(s)ds \underbrace{F(t-\tau)}_{F(t-s)F(s-\tau)}\\
		& = 2  \int_{t=s}^\infty \int_{s=\tau}^t h(t) F(t-s)h(s)ds\underbrace{\int_{\tau = 0}^{s} d\tau i(\tau)F(s-\tau)}_{y(s)}\\
		& = 2  \int_{t=0}^\infty \int_{s=0}^t h(t)ds F(t-s)\underbrace{h(s) y(s)}_{i(s)}\\
		& = 2  \int_{t=0}^\infty  h(t) \underbrace{\int_{s=0}^t ds F(t-s){i(s)}}_{y(t)}\\
		& = 2 \underbrace{\int_{t=0}^\infty i(t) dt}_{Z} \\
		& = 2Z \label{eq_C2final}
		\end{align}
		\end{subequations}
	\end{small}
\subsection*{Numerical Simulations}
For simulation, we consider the following compartmental model
\begin{subequations} \label{eq_Erlang}
	\begin{align}
	\dot{x} &= -\beta(t) x^{\kappa + 1}y + m\sigma z ,\\
	\dot{y}_1 &= \beta(t) x^{\kappa +1}y - my_1,\\
	& \vdots\\
	\dot{y}_m &= my_{m-1} - m y_m,\\
	\dot{z} &= my_m - m\sigma z,
	\end{align}
\end{subequations}
where 
$\beta(t) = \beta_0\exp(\alpha\sin(\omega t))$.

The incidence for the model in \Cref{eq_Erlang} is defined by 
$$i(t) = \beta(t) x^{\kappa + 1}(t) y(t),$$
and the expected case reproductive number of individuals got infected at time $\tau$ and recovered at time $\rho$ is formulated as 
$$C(\tau, \rho) = \int_{\tau}^{\rho} \beta(t) x^{\kappa +1} dt.$$
\subsection*{Simulation Results for Seasonally Forced Epidemics}
\begin{figure}[H] 
\includegraphics[width=1\linewidth]{figs/timePlot.Rout.pdf}
\caption{A seasonally forced epidemic with a period of $60$ (\Cref{eq_Erlang} with $m=1$ and $\kappa=0$).
		The simulation started at $y(0)=10^{-9}$ and was run for $200$ time units.
	Panels e and d depict the mean and variance in cohort's case reproductive number, respectively. 
	The transmission rate is periodic with a two-month period (panel g).
The instantaneous reproductive number is plotted in panel f.
	Cohort statistics were plotted over the time interval $[100,160]$ (Panels d,e,f). 
}
\label{fig_seasonally}
\end{figure}
\begin{figure}[H] 
	\centering
	\includegraphics[width=1.35\linewidth]{figs/RcbarPlotVaryingSigma.Rout.pdf}
	\caption{Variance in the case reproductive number remains constant regardless of  average transmission rate ($\beta_0 = 1.5,3.5,5,8$) and immunity waning rate ($\sigma=0.02,0.05,0.1$). 
		The model has one infectious compartment and is linear in transmission (\Cref{eq_Erlang}, $m=1$, $\kappa=0$).
		The period of transmission rate is set to $60$ ($\omega = \tau/60$, $\alpha=1$).
		The panels on the left depict the between and within variances in $R_c$; weighted by incidence (a) and averaged over time (c).
		The panels on the right depict the total variances in $R_c$, computed independently from the corresponding panels on the left: weighted by incidence (b) and averaged over time (d).
		The simulation started at $y(0)=10^{-9}$ and was run for $200$ time units.
		Cohort statistics were calculated over the time interval $[100,160]$. 
		\Cref{fig_seasonally} depicts the time evolution of the system for $\sigma=0.1.$
	}
\label{fig_Rc_seasonally}
\end{figure}
\subsection*{Simulation Results for Nonlinear Incidence}
\begin{figure}[H]
	\includegraphics[width=1.35\linewidth, trim=0cm 0cm 0cm 1cm, clip]{figs/RcbarPlotVaryingKappa.Rout.pdf}
	\caption{Variance in $R_c$ for an SIR model with nonlinear transmission (\Cref{eq_Erlang} with $m=1$, $\sigma=0$, $\kappa=0, 0.5, 1$, and $\omega=0$).
		Top panels report the variance in $R_c$ weighted by incidence, while panels at the bottom report those weighted by time.
		The duration of simulation is $100$ time units, and the initial condition is $y(0)=10^{-9}.$
	}
\end{figure}
\newpage
	\subsection{Higher moments of the case reproductive number}
\begin{align*}
C(\tau, \rho) &= \beta \int_{t = \tau}^{t= \rho} x(t) dt,\\
\omega (\tau, \rho) & = i(\tau) f(\rho - \tau),\\
i(t) & = \beta x(t) y(t),
\end{align*}
where $x(t)$ and
$y(t)$  are the proportions of susceptible and infectious individuals, respectively, 
$\omega(\tau, \rho)$ is the size of individuals infected at time point $\tau$ and recovered at time point $\rho$,
$f(t)$ is the distribution of residence time in the infectious compartment, and $i(t)$ is the incidence at time point $t$.


The $k^\text{th}$ raw moment of the expected case reproductive number $C(\tau,\rho)$ reads as
	\begin{align*} 
	C_k &= \int \int i(\tau) f(\rho-\tau) d\tau d\rho \big (\int_{\tau}^{\rho} \beta x(s) ds \big )^k \\
	&= \beta^k \int \int i(\tau) f(\rho-\tau) d\tau d\rho \underbrace{\int_{\tau}^{\rho} \cdots \int_{\tau}^{\rho}}_{k \text{ times}} x(t_1) \ldots x(t_k) dt_1 \ldots dt_k
	\end{align*}
	The integral of the symmetric $k$-variable function $x(t_1)\ldots x(t_k)$  over the hybercube $[\tau, \rho]^k$ is equal to $k!$ times the integral of the function over the region $\tau<t_1<t_2<\ldots<t_k < \rho$.
	Hence, we have
	\begin{align*}
	&= k! \beta^k \int_{\tau<t_1<t_2<\ldots<t_k < \rho} \int \underbrace{\int \ldots \int}_{k \text{ times}}
	i(\tau) f(\rho-\tau) d\rho d\tau x(t_1) \ldots x(t_k) dt_1 \ldots dt_k  \\
	&= k! \beta^k \int_{\tau<t_1<t_2<\ldots<t_k } \underbrace{\int \ldots \int}_{k \text{ times}}
	i(\tau)  d\tau x(t_1) \ldots x(t_k) dt_1 \ldots dt_k  \underbrace{\int_{\rho = t_k}^\infty f(\rho - \tau) d\rho}_{F(t_k-\tau)}
	\end{align*}
	Using the Markovian property, we have
	\(F(t_k - \tau) = F(t_k - t_{k-1})\times F(t_{k-1} - t_{k-2}) \times \ldots \times F(t_1 - \tau) \).
	By defining $t_0$ as $\tau$,
	the term $ \int_{t_l<t_{l+1}} dt_l i(t_l) F(t_{l+1} - t_l)$, for $l \in \{0,\ldots,k-1\}$, is equal to $y(t_{l+1})$.
	The multiplication of $y(t_{l+1})$ and $\beta x(t_{l+1})$ equals $i(t_{l+1}).$ 
	By repeating the same procedure for $k$ times,
	the remaining element would be $k! \int i(t_k) dt_k$.
	Hence, the $k$th normalized raw moment, defined as $C_k/Z$ is equal to $k!.$
	
	In other words, the distribution of the case reproductive number is exponential with rate $1$.
	
	We calculated the third and fourth normalized moments of $R_c$ for a simple SIR model.
\begin{figure}[H]
	\centering
	\includegraphics[width=1.2\linewidth]{figs/ErlangMoments.Rout.pdf}
	\caption{The mean, variance, third, and  fourth normalized raw moments of $R_c$.
		Simulation results support the theory that the third and fourth normalized raw moments are equal to $3!=6$ and $4!=24$ regardless of the value for $\beta_0$ in a simple SIR model $(m=1)$.
	However, it is not the case for Erlang distribution $(m=2,4)$.
	$\alpha$ and $\sigma$ are set to zero, and the duration of the simulation is $100$ time units, starting from the initial condition $y(0)=10^{-9}$.}
\end{figure}
\subsection{Erlang infectious periods don't allow variance	above one}
A distribution function $f(t)$ with survival function $F(t)$ and non-negative support is said to be in increasing failure rate class if the corresponding hazard function, $f(t)/F(t)$, is increasing, or, equivalently, $\ln F(t)$ is concave.
Erlang distribution, for example, belongs to this class \cite{leemis2023statistical}.
This implies the following property \cite{marshall1972classes}
\begin{equation}
F(a +b) < F(a) F(b). \label{eq_submultiplicative}
\end{equation}
Given \eqref{eq_CaseReproductiveNumber} and in view of  \eqref{eq_before_Markovian}, the second raw moment of $C(\tau, \rho)$
equals 
\begin{align*}
C_2 &= 2 \int_{\tau = 0}^\infty \int_{t=\tau}^\infty \int_{s=\tau}^t i(\tau)h(t)h(s)ds F(t-\tau)\\
&\overset{\eqref{eq_submultiplicative}}{<} 2 \int_{\tau = 0}^\infty \int_{t=\tau}^\infty \int_{s=\tau}^t i(\tau)h(t)h(s)ds F(t-s)F(s-\tau)\\
\xrightarrow{\eqref{eq_C2final}} & < 2Z.
\end{align*}
Previously it has been shown that the mean of the expected case reproductive number, regardless of the infectious period distribution, is one yielding
$$
\frac{C_2}{Z} - 1 < 1.
$$
\subsubsection*{Simulations for Erlang Distribution of Infectious Period}
\begin{figure}[H]
	\includegraphics[width=1.2\linewidth]{figs/bars.Rout.pdf}
	\caption{Variance in the case reproductive number $R_c$ is less than one in the presence of Erlang distribution of infectious period.
		The period of transmission rate is $T=60$ ($\omega = \tau/6, \alpha = 1$).
		The rate of waning immunity $\sigma$ is set to $0.1$.
		$\kappa$ is set to zero.
		Other simulation details are the same as those in \Cref{fig_Rc_seasonally}.
	}
\end{figure}
%\begin{figure}[H]
%	\includegraphics[width=1.2\linewidth]{figs/ErlangMoments.Rout.pdf}
%	\caption{The mean, variance, and normalized higher raw moments of the case reproductive number $R_c$ in the presence of Erlang distribution of infectious period $(m=2,4)$.
%	}
%\end{figure}
	\section{Instantaneous case reproductive number}
For the following SIR model,
\begin{align*}
\dot{x} &= - \beta(t)yx,\\
\dot{y} &= \beta(t) yx -y,
\end{align*}
we define the instantaneous case reproductive number associated with cohorts infected at time point $\tau$ as 
\[R_i(\tau) = \beta(\tau)x(\tau).\]
%Zeroth moment of $R_i$ is defined as
%\begin{align*}
%R_0 
%&= \int_{\tau} i(\tau) \\
%&= Z.
%\end{align*}
%The first raw moment of $R_i$ is then equal to 
%\begin{align*}
%R_1 &= \int_{\tau}d\tau i(\tau)\beta(\tau) x(\tau)
%\end{align*}
%The second raw moment of $R_i$ is equal to 
%\begin{align*}
%R_2 &= \int_{\tau} d\tau  i(\tau)  \beta^2(\tau) x^2(\tau) 
%\end{align*}
\begin{figure}[H]
	\centering
	\includegraphics[width=1\linewidth]{figs/RiMoments.Rout.pdf}
	\caption{The mean, variance, and normalized third raw moment of instantaneous reproductive number for different number of infectious compartments $(m=1,2,4)$, and transmission rates $(\beta_0 = 1.5,5,8)$.
		$\alpha$ and $\sigma$ are set to zero, and the duration of the simulation is $100$ time units, starting from the initial condition $y(0)=10^{-9}$.
		The number of steps is $10^4.$
		Panel a shows the  incidence-weighted mean of $R_i$ ($\mu = \frac{\int R_i(\tau) i(\tau) d\tau}{\int i(\tau)d\tau}$).
		Panel b reports the variance in $R_i$ ($\frac{\int R_i^2(\tau) i(\tau) d\tau}{\int i(
			\tau)d\tau} - \mu^2$).
		Panel c reports the normalized third raw moment $R_i$ ($\frac{\int R_i^3(\tau) i(\tau) d\tau}{\int i(
			\tau)d\tau}$).
		Panel d show the time-averaged mean of $R_i$ ($\mu_t = \frac{\int R_i d\tau}{\int d\tau}$), and panel e reports the variance in $R_i$ averaged by time ($\frac{\int R_i^2(\tau) d\tau}{\int d\tau} - \mu_t^2$). }
\end{figure}
\begin{figure}[H]
	\centering
	\includegraphics[width=1.1\linewidth]{figs/RiHigherMoments.Rout.pdf}
	\caption{
		The mean, variance, and normalized third and fourth raw moments of the instantaneous reproductive number
		are plotted against the corresponding quantities for the Uniform distribution $U(\beta_0x(\infty),\beta_0x(0))$
		 for different values of transmission rates $(\beta_0 = 1.2,2,4,8)$ in a simple SIR model ($m=1$).
		$\alpha$ and $\sigma$ are set to zero, and the duration of the simulation is $100$ time units, starting from the initial condition $y(0)=10^{-9}$.
		The number of steps is $3000.$	 }
\end{figure}
%\begin{figure}[H]
%	\centering
%	\includegraphics[width=0.8\linewidth]{Codes/plots/barPlotRi.pdf}
%	\caption{Calculated variance in the instantaneous reproductive number for different seasonally forced periods ($\omega = 0, \tau/60\approx 0.105, \tau/20 \approx 0.314$) and average transmission rates ($\beta_0 = 1.5,2,3,4,5$). 
%		The instantaneous reproductive number ($R_i(\tau) = \beta(\tau)x(\tau)$)
%		is shared among the individuals in the same cohort share, and, accordingly, the calculated variance stems from the between-cohort variance.
%		The model has one infectious compartment (\Cref{eq_Erlang}, $m=1$), and the immunity waning rate is set to 1 (\Cref{eq_Erlang}, $\sigma=1$).
%		The length of simulation is $365$. 
%	}
%\end{figure}
\begin{figure}[H]
	\centering
	\includegraphics[width=1\linewidth]{figs/RibarPlotVaryingSigma.Rout.pdf}
	\caption{Calculated variance in the instantaneous reproductive number for different rates of waning immunity ($\sigma=0.02,0.05,0.1$) and average transmission rates ($\beta_0 = 1.5,3.5,5,8$). 
		The model has one infectious compartment (\Cref{eq_Erlang}, $m=1$).
		The top panel depicts the variance weighted by the incidence, and the panel at the bottom depicts the variance obtained by averaging over time.
		The variance was computed for the duration $[100,160]$.
		Transmission rate has a period of $T= 60$ days ($\omega = \tau/60$, $\alpha=1$).
		Other simulation details are the same as those in \Cref{fig_Rc_seasonally}.
	}
\end{figure}
%\begin{figure}
%	\centering
%	\includegraphics[width=0.8\linewidth]{Codes/plots/barPlotRiNonlinear.pdf}
%	\caption{Calculated variance in the instantaneous reproductive number for  nonlinear transmission ($\kappa = 0, 0.5, 1$) and  transmission rates ($\beta_0 = 1.5,2,3,4,5$). 
%		The instantaneous reproductive number ($R_i(\tau) = \beta(\tau)x^\kappa(\tau)$)
%		is shared among the individuals in the same cohort share, and, accordingly, the calculated variance stems from the between-cohort variance.
%		The model has one infectious compartment (\Cref{eq_Erlang}, $m=1$).
%		The duration of simulation is $365$.
%	}
%\end{figure}
\newpage
\section{Intrinsic generation-interval distribution and infectious period distribution}
The intrinsic generation-interval distribution, for a constant transmission rate, $g(t)$ is defined as 
\begin{equation}
	g(t) = \frac{F(t)}{\int F(t) dt},
\end{equation}
where $F(t)$ is the survival function corresponding to the infectious period distribution \cite{champredon2015intrinsic}.

According to \cite[page 150]{feller} and \cite{chakraborti}, for a continuous non-negative random variable $t$ with survival function $F(t)$, the following relation holds:
\begin{equation} \label{eq_survial_higher}
	E(t^r) = \int_0^\infty r\tau^{r-1}F(\tau)d\tau, \quad r\geq 1.
\end{equation}
We then have
\begin{subequations}
\begin{align}
 t^2 g(t) & = \frac{t^2F(t)}{\int F(t) dt}\\
  \int t^2 g(t) dt & = \frac{ \int t^2F(t) dt}{\int F(t) dt}\\
  \int t^2 g(t) dt & = \frac{E(t^3)}{3\int F(t) dt} \text{ (Using \Cref{eq_survial_higher}) } \\
  \int t^2 g(t) dt & = \frac{E(t^3)}{3E(t)} \text{ (Using \Cref{eq_survial_higher})}\\
   V_g + \mu^2_g & = \frac{E(t^3)}{3\mu_i} \\
    \kappa_g + 1 & = \frac{E(t^3)}{3\mu_i \mu^2_g} \\
         \kappa_g & = \frac{E(t^3)}{3\mu_i \mu^2_g} -1 \label{eq_cvg}
\end{align}
\end{subequations}
The means $\mu_i$ and $\mu_g$ are related by
\begin{equation} \label{eq_mug_mui}
	\mu_g = (1 + \kappa_i)\frac{\mu_i}{2}
\end{equation}
Substituting \Cref{eq_mug_mui} to \Cref{eq_cvg} results in
\begin{equation} \label{eq_cv2g}
 \kappa_g  = \frac{4E(t^3)}{3\mu_i^3 (1 + \kappa_i)^2} -1.
\end{equation}
%Defining coefficient of skewness, $\gamma_i$, as $E\big((\frac{t-\mu_i}{\sigma_i})^3\big)$, or, equivalently
%\begin{equation}
%	\gamma_i = \frac{E(t^3) - 3\mu_i \sigma^2_i - \mu^3_i}{\sigma^3_i},
%\end{equation}
%\Cref{eq_cv2g} can be written in terms of skewness and coefficient of variation:
%\begin{equation}
%\kappa_g  =	\frac{4\gamma_i\kappa^{3/2}_i + 12\kappa_i + 4}{3(1+\kappa_i)^2}-1
%\end{equation}
%\subsection{Using ration of means $\operatorname{PM}$ to simplify \Cref{eq_cv2g}}
%By defining
%\begin{align}
%	\operatorname{PM}_x(l+1) = \frac{\int x^{l+1}f(x) dx}{\int x^{l}f(x) dx}, \quad \text{for } l \in \mathbb{Z}_{\geq 0},
%\end{align}
%we have $\operatorname{PM}_x(2) = \mu_x(1 + \kappa_x)$ and
%\begin{align*}
%\operatorname{PM}_x(3) &= \frac{\int x^{3}f(x) dx}{\int x^2f(x) dx}\\
%& = \frac{E(x^3)}{\mu^2_x(1 + \kappa_x)}.
%\end{align*}
%  \Cref{eq_cv2g} can be rewritten as follows:
%\begin{align}
%	\kappa_g = \frac{4}{3}\frac{\operatorname{PM}_i(3)}{\operatorname{PM}_i(2)} -1.
%\end{align}
%Moving $1$ to the left-hand side and replacing $1+\kappa_g$ by 
%$\operatorname{PM}_g(2)/\operatorname{PM}_g(1)$
%results in the equation:
%\begin{align}
%\frac{\operatorname{PM}_g(2)}{\operatorname{PM}_g(1)} = \frac{4}{3}\frac{\operatorname{PM}_i(3)}{\operatorname{PM}_i(2)}.
%\end{align}
\subsection{A new measure for skewness}
We define the coefficient $\phi$ as
\begin{align}
	\phi = \frac{\operatorname{E}(t^3)}{\mu^3 (1+\kappa)^2} -1,
\end{align}
where $\mu$ is the mean and $\kappa$ is the square of the coefficient of variation.
The third raw moment can be expressed in terms of $\mu$, $\kappa$, and $\phi$ as follows:
\begin{align} \label{eq_phi}
	\operatorname{E}(t^3) = \mu^3 (1+\kappa)^2 (1+\phi),
\end{align}
Substituting \Cref{eq_phi} for $\operatorname{E}(t^3)$ in \Cref{eq_cv2g} results in
\begin{equation}
	\kappa_g = \frac{1 + 4 \phi_i}{3}.
\end{equation}

For the gamma distribution, $\phi$ is equal to $\frac{\kappa}{\kappa + 1},$ and for the log-normal distribution, $\phi$ equals $\kappa$.
The calculation for the log-normal distribution is as follows:

For $\ln(x) \sim \mathcal{N}(M, S)$, the mean, variance, and the third raw moments are
\begin{align*}
	\mu & = \exp(M + S^2/2),\\ 
	\sigma^2 & = \exp(2M + S^2)\big(\exp(S^2) -1\big),\\
	\operatorname{E}(x^3)& = \exp(3M + \frac{9S^2}{2}).\\
\end{align*}
The square of coefficient of variation $\kappa$ is then
\begin{align*}
\kappa &= \frac{\sigma^2}{\mu^2}\\
 &= \frac{\exp(2M + S^2)\big(\exp(S^2) -1\big)}{\exp(2M + S^2)}\\
 & = \exp(S^2)-1.
\end{align*}
Substituting $\operatorname{E}(x^3)$, 
$\mu$, and $\kappa$ in \Cref{eq_phi} results in 
\begin{align*}
\exp(3M + \frac{9S^2}{2}) &= \big(\exp(M + S^2/2) \big)^3 \big(1 +\exp(S^2)-1 )^2 (1 + \phi)\\
 &= \exp(3M + 3S^2/2)\exp(2S^2)(1 + \phi)
\end{align*}
Dividing both sides by $\exp(3M + 3S^2/2)\exp(2S^2)$ results in 
$
	\exp(S^2) -1 = \phi,
$
or, equivalently, 
\begin{equation}
\phi = \kappa.
\end{equation}
\subsection{Special case: Erlang distribution}
If the infectious period is Erlang distributed with rate $\lambda$ and shape $1/\kappa$, then according to \cite[page 70]{walck2007hand}, we have
\begin{equation}
	E(t^n) = \frac{1}{\lambda^n} \prod_{m=0}^{n-1} (\frac{1}{\kappa} + m),
\end{equation}
which yields
\begin{equation}
	E(t^3) = \frac{1}{\lambda^3}\frac{1}{\kappa}(\frac{1}{\kappa}+1)(\frac{1}{\kappa}+2),
\end{equation}
and given $\mu_i = \frac{1}{\kappa\lambda}$, and $\text{CV}^2_i = \kappa$,
 \Cref{eq_cv2g} simplifies to
\begin{align*}
\text{CV}^2_g  &= \frac{4(\frac{1}{\kappa}+2)}{3(\frac{1}{\kappa}+1)} -1\\
& = \frac{1 + 5\kappa}{3(\kappa +1)}
\end{align*}
\newpage
\bibliographystyle{plain}
\bibliography{ref}
\end{document}
